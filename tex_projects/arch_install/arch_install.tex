\documentclass{article} 
\usepackage[T1]{fontenc}
\usepackage[utf8]{inputenc}

\usepackage{newtxtext,newtxmath}
\usepackage{microtype}
\usepackage{amsmath,amsfonts,amssymb}
\usepackage{microtype}

\usepackage{hyperref}
\usepackage{bookmark}

\begin{document}

I, William C. Dawn, originally prepared this document for configuring my Lenovo
ThinkPad x220 experiment(s). Information is also provided relating to
``\verb|dawnsvr|'' which was some sort of HP monstrosity.

This is a living document. I update it as I learn more information. At any point
run \verb|lsblk| to see the status of block drives on the system.

\section{Make Arch Bootable USB}
  \begin{enumerate}
    \item Download latest Arch Linux iso\\
      https://www.archlinux.org/download/
    \item Do \emph{NOT} mount the drive. It is also a good idea to format the
      drive to FAT/VFAT.
    \item Use \verb|dd| (``Data Dump'' or ``Disk Destroyer'') to make the arch 
      live USB. If \verb|/dev/sdb| is the usb drive:
\begin{verbatim}
$ sudo dd if=/path/to/archlinux.iso of=/dev/sdb status="progress"
\end{verbatim}
  \end{enumerate}

\section{Boot from Live USB.}
  \begin{itemize}
    \item Ethernet will need to be connected at boot.
    \item Go ahead and boot from the Live USB. Use F12 on the Lenovo ThinkPad
      x220.
    \item It's a good idea to check ping to make sure you have internet because
      you'll need it.
    \item Hook up an ethernet cable.
\begin{verbatim}
# ping archlinux.org
\end{verbatim}
    \item If the ping fails, run \verb|dhcpcd| to get an ip address.
  \end{itemize}

\section{BIOS Notes.}
  \begin{itemize}
    \item Much of the same applies for a BIOS only system (like the old HP
      desktop machine I use called \verb|dawnsvr|).
    \item The differences are best summarized on the Arch Linux wiki: ``A BIOS
      boot partition is only required when using GRUB for BIOS booting from a
      GPT disk. The partition has nothing to do with \verb|/boot| and it must
      not be formatted with a filesystem or mounted.''
    \item The following are exceptions:
    \begin{itemize}
      \item Do \textbf{NOT} format \verb|/dev/sda1|. The BIOS partition must
        contain no filesystem.
      \item Do \textbf{NOT} mount \verb|/dev/sda1|.
      \item When it comes to install a boot loader.
\begin{verbatim}
# grub-install /dev/sda
# grub-mkconfig -o /boot/grub/grub.cfg
\end{verbatim}
      \item And that should do it...
    \end{itemize}
  \end{itemize}


\section{Make Drive Partitions.}
  \begin{itemize}
    \item We will assume four partitions.
    \begin{enumerate}
      \item \verb|/boot| (boot for efi)
      \item \verb|SWAP| to allow for system suspend.
      \begin{itemize}
        \item \verb|SWAP| partition must be $1.5\times$ RAM (e.g.~12GB)
        \item This is necessary to allow all RAM to write to disk for suspend.
      \end{itemize}
      \item \verb|/| (root)
      \item \verb|/home| (home) to allow to reinstall root if need be.
    \end{enumerate}
    \item This also assumes that we're installing on a bare-metal device with
      UEFI hardware. This is based on the Lenovo ThinkPad x220 that I bought on
      eBay.
    \item Use \verb|gdisk| to accomplish this. Run: 
\begin{verbatim}
# gdisk /dev/sda
\end{verbatim}
      If \verb|/dev/sda| is the main hard drive disk.
    \item If the drive has partitions on it, it is first necessary to delete
      existing partitions. This can be done within \verb|gdisk|.
    \item When you run \verb|gdisk| you'll be led through a series of prompts.
      Hypothetically, these are some recommended partition sizes and drive types
      (specified within \verb|gdisk|).
    \begin{enumerate}
      \item \verb|/boot|, 1 GB, EFI System
      \item \verb|/|, 30 GB, Linux System
      \item \verb|SWAP|, 12 GB, Linux Swap
      \item \verb|/home|, remainder, Linux System
    \end{enumerate}
\end{itemize}

\section{Format the Partitions.}
\begin{itemize}
  \item Drives are formatted to the default linux file system \verb|ext4|.
    Except for the UEFI \verb|/boot| partition which will be formatted in
    \verb|FAT|.
  \item Don't sweat file systems too much. It's a waste of time.
  \item If you do want to waste time, some people seem to think \verb|Btrfs|
    is better. It is certainly newer, but it is also less stable. It does not do
    journaling so it decreases the number of write-to-disk operations.
  \item \verb|SWAP| partition must be established specially.
  \item Format each of the partitions as follows:
  \begin{enumerate}
    \item \verb|/boot| \textbf{NOTE} The efi partition must be FAT.
  \end{enumerate}
\begin{verbatim}
# mkfs.vfat /dev/sda1
\end{verbatim}
  \begin{enumerate}
    \item \verb|/|
  \end{enumerate}
\begin{verbatim}
# mkfs.ext4 /dev/sda2
\end{verbatim}
  \begin{enumerate}
    \item \verb|SWAP|
  \end{enumerate}
\begin{verbatim}
# mkswap /dev/sda3
# swapon /dev/sda3
\end{verbatim}
  \begin{enumerate}
    \item \verb|/home|
  \end{enumerate}
\begin{verbatim}
# mkfs.ext4 /dev/sda4
\end{verbatim}
\end{itemize}

\section{Mount the Partitions.}
\begin{itemize}
  \item Now we have made and formatted the partitions. We're going to mount them
    at \verb|/mnt| because that's what people do.
  \item Intermediately, I have to pause to make directories where I can
    subsequently mount drives.
  \begin{enumerate}
    \item \verb|/|
  \end{enumerate}
\begin{verbatim}
# mount /dev/sda2 /mnt
\end{verbatim}
  \begin{enumerate}
    \item \verb|/boot|
  \end{enumerate}
\begin{verbatim}
# mkdir /mnt/boot
# mount /dev/sda1 /mnt/boot
\end{verbatim}
  \begin{enumerate}
    \item \verb|/home|
  \end{enumerate}
\begin{verbatim}
# mkdir /mnt/home
# mount /dev/sda4 /mnt/home
\end{verbatim}
\end{itemize}

\section{Install Arch Linux.}
  \begin{itemize}
    \item This is done using the \verb|pacstrap| command.
    \item Install \verb|base|, \verb|base-devel|, and \verb|vim| packages
      because we'll probably need them all anyway. We also need
      \verb|intel-ucode| or else we can't boot.
\begin{verbatim}
# pacstrap /mnt base base-devel vim intel-ucode
\end{verbatim}
  \end{itemize}

\section{Make fstab (File System TABle)}
  \begin{itemize}
    \item We want the \verb|-U| option to use UUIDs.
\begin{verbatim}
# genfstab -U /mnt > /mnt/etc/fstab
\end{verbatim}
    \item You need to check after the command runs to make sure it did what you
      thought it was going to do.
\begin{verbatim}
# vim /mnt/etc/fstab
\end{verbatim}
    \item Use set \verb|noatime| for root (\verb|/|) and home
      (\verb|/home|). \verb|noatime| decreases the number of times data is
      written to a disk and the SSD can benefit. It prevents the last time a
      file was opened from being recorded. An example from the arch wiki is
      provided.
  \end{itemize}
\begin{verbatim}
# <device>   <dir>   <type>   <options>   <dump> <fsck>
/dev/sda1    /boot   vfat     defaults    0      1
/dev/sda2    /       ext4     noatime     0      1
/dev/sda3    none    swap     defaults    0      0
/dev/sda4    /home   ext4     noatime     0      2
\end{verbatim}

\section{Change root.}
  \begin{itemize}
    \item This is when we (mostly) have an operating system.
\begin{verbatim}
# arch-chroot /mnt
\end{verbatim}
    \item Now everything is on the main computer/hard drive.
  \end{itemize}

\section{Download and Configure Network Manager.}
  \begin{itemize}
    \item Use \verb|networkmanager| for network management and WiFi. It will
      work with eduroam.
    \item Connman may seem like a good idea but it's not as robust.
    \item The commandline command is \verb|nmcli|. There is also a text-based
      interface via \verb|nmtui|.
    \item Download the package.  
\begin{verbatim}
# pacman -S networkmanager
\end{verbatim}
    \item Start with systemd. Note the capitalization.
\begin{verbatim}
# systemctl enable NeworkManager
\end{verbatim}
  \end{itemize}

\section{Set TRIM settings for SSD.}
  \begin{itemize}
    \item For a SSD, TRIM support can significantly improve timing by keeping
      the disk in a state that is always ready to be written.
    \item This is done periodically, once per week, (\emph{not} continuously) by
      a systemd task.
    \item Simply execute this to kick the whole thing off.
  \end{itemize}
\begin{verbatim}
# systemctl enable fstrim.timer
\end{verbatim}

\section{Install a Boot Loader/Manager.}
  \begin{itemize}
    \item Traditionally, a bootloader would be used that would start the OS. Now
      a days, OSes can typically start themselves and just need to be told to do
      so via a bootmanager.
    \item \verb|systemd-boot| is the most lightweight bootmanager on the
      market it seems.
    \item If this doesn't work, use \verb|grub|.
    \item Command should be installed in the Arch ``base'' package group.
\begin{verbatim}
# bootctl --path=/boot install
\end{verbatim}
    \item It seems like you need to generate your own config files at
      \verb|/boot/loader/loader.conf| and
        \verb|/boot/loader/entries/arch.conf|.
    \item The loader config at \verb|/boot/loader/loader.conf| should contain
\begin{verbatim}
default arch
timeout 0
editor 0
editor yes
\end{verbatim}
    \item The default profile at \verb|/boot/loader/entries/arch.conf|
      (corresponding to default name) should contain
\begin{verbatim}
title Arch Linux
linux /vmlinuz-linux
initrd /intel-ucode.img
initrd /initramfs-linux.img
options root=PARTUUID=###### rw
\end{verbatim}
    \item Generate the PARTUUID with the command
\begin{verbatim}
# blkid -s PARTUUID -o value /dev/sda2
\end{verbatim}
      To dump this straight into vim, use
      \verb|:read !<shell>| A hook must be placed
      to allow pacman to automatically update systemd-boot. Otherwise a
      special command is required (and I'd forget). A file at
      \verb|/etc/pacman.d/hooks/100-systemd-boot.hook| should contain
\begin{verbatim}
[Trigger]
Type = Package
Operation = Upgrade
Target = systemd

[Action]
Description = Updating systemd-boot
When = PostTransaction
Exec = /usr/bin/bootctl update
\end{verbatim}
  \end{itemize}

\subsection{Set Root Password.}
  \begin{itemize}
    \item Use the \verb|passwd| command.
\begin{verbatim}
# passwd
\end{verbatim}
  \end{itemize}

\section{Set Locale Information.}
  \begin{itemize}
    \item Set information necessary for timezone and language and character
      sets.
    \item Set the time zone. Use tab complete to make a symlink.
\begin{verbatim}
# ln -sf /usr/share/zoneinfo/America/CITY /etc/localtime
# hwclock --systohc
\end{verbatim}
    \item Editing the file \verb|/etc/locale.gen|. Uncomment
      \verb|en_US.UTF-8 UTF-8| and \verb|en_US ISO-8859-1|.
    \item Run the command to generate the locale file.
\begin{verbatim}
# locale-gen
\end{verbatim}
    \item Set the \verb|LANG| variable in \verb|/etc/locale.conf|.
\begin{verbatim}
LANG=en_US.UTF-8
\end{verbatim}
  \end{itemize}

\section{Set Hostname.}
  \begin{itemize}
    \item Edit the value of the file in \verb|/etc/hostname|.
    \item For now, I'm thinking \verb|archpad|.
  \end{itemize}

\section{Unmount and Reboot.}
  \begin{itemize}
    \item Everything (should be) (is) done.
    \item Exit the new install back onto the Live USB.
\begin{verbatim}
# exit
\end{verbatim}
  \item Unmount all of the drives.
\begin{verbatim}
# swapoff /dev/sda3
# umount -R /mnt
\end{verbatim}
  \item Reboot.
\begin{verbatim}
# reboot
\end{verbatim}
    \item You should get dropped back in to the tty.
  \end{itemize}

\section{Post-Install Configuration.}
  \subsection{Connect to WiFi.}
    \begin{itemize}
      \item This seemed flaky\ldots But, it seems you just have to start this.
\begin{verbatim}
# nmcli device wifi connect NETGEAR35 password MYPASSWORD
\end{verbatim}
      \item \textbf{\emph{NOTE}} the password will be stored in plain text.
      \item This didn't work to automatically connect the first time I tried it.
        The only way I found to fix it was to disable and delete the connection
        named \verb|NETGEAR35| and then run the above command a second time.
    \end{itemize}

  \subsection{Make a user.}
    \begin{itemize}
      \item You'll want a user that's not root. Just a good idea.
      \item Think of the \verb|wheel| group as the administrator group.
\begin{verbatim}
# useradd -m -g wheel wcdawn
# passwd wcdawn
\end{verbatim}
      \item Edit the sudoers file. Users of the wheel group should be allowed to
        run any command without a password. Uncomment the line
\begin{verbatim}
%wheel ALL=(ALL) NOPASSWD: ALL
\end{verbatim}
    \end{itemize}

  \subsection{Hibernate/Suspend.}
    \begin{itemize}
      \item It looks like systemd should do this automatically.
      \item If it doesn't, look in the config file
        \verb|/etc/systemd/logind.conf| and grep the line
        \verb|HandleLidSwitch|.
      \item You'll probably need to uncomment the lines relating to
        \verb|HandleLidSwitch|.
    \end{itemize}

  \subsection{Setup Graphical Environment.}
    \begin{itemize}
      \item Install the graphics and TrackPoint drivers for the ThinkPad x220.
\begin{verbatim}
# pacman -S xf86-video-intel xf86-input-libinput
\end{verbatim}
      \item Install the X.org packages using pacman.
\begin{verbatim}
# pacman -S xorg-server xorg-xinit
\end{verbatim}
      \item Install i3-gaps.
\begin{verbatim}
# pacman -S i3-gaps compton feh i3blocks dmenu ttf-ibm-plex
\end{verbatim}
      \item Configuration of Window Managers are placed in
        \verb|~/.xinitrc|.
      \item Setup xinit config. In \verb|~/.xinitrc| add the
        line \verb|exec i3|.
      \item Type \verb|xinit| into the tty.
      \item Next time you login, use \verb|startx|. Probably add
        \verb|startx| to bashrc.
\begin{verbatim}
if [[ "$(tty)" = "/dev/tty1" ]]
then
    pgrep i3 || startx
fi
\end{verbatim}
      \item At this point, it's probably a good idea to try to clone my dotfiles
        repo.
      \item You'll want to use \verb|~/scripts/install_st.sh| to install and configure suckless terminal.
    \end{itemize}

  \subsection{Setup Audio.}
    You'll want to use the \verb|Alsa| package to manage audio.
\begin{verbatim}
$ sudo pacman -S alsa-utils
\end{verbatim}
    Then, run the manager program to set the audio levels.
\begin{verbatim}
$ alsamixer
\end{verbatim}
    Type \verb|M| to unmute the master channel and then use the arrow keys to
    set the volume. Type \verb|Esc| to exit. You can then use 
\begin{verbatim}
$ speaker-test -c2
\end{verbatim}
    to perform a two channel audio test.

  \subsection{Setup \LaTeX.}
    \begin{itemize}
      \item You gotta have it\ldots{} I'm working on my thesis right now.
      \item Don't use mupdf. Use poppler instead. I had weird color rendering
        problems and segfaults with mupdf.
\begin{verbatim}
# pacman -S texlive-most zathura biber zathura-pdf-poppler ghostscript
\end{verbatim}
    \end{itemize}

  \subsection{Numeric/Computational Packages}
    \begin{itemize}
      \item TODO need to build some sort of list of packages.
      \item Install gfortran (it's not bundled with the default gcc) and lapack.
\begin{verbatim}
# pacman -S gcc-fortran lapack
\end{verbatim}
    \end{itemize}

  \subsection{Install a web browser}
    \begin{itemize}
      \item For now, I've selected chromium, an open-source version of Google
        chrome.
      \item This is a bit tough becasue there are a few packages involved and
        one of them is in the Arch User Repository (AUR).
      \item Install the main chromium package.
\begin{verbatim}
# pacman -S chromium
\end{verbatim}
      \item If you need flash (you probably don't), install that too.
\begin{verbatim}
# pacman -S pepper-flash
\end{verbatim}
      \item Now, you need a package from the AUR. This probably deserves a
        writeup of it's own.
    \end{itemize}

    \subsubsection{Installing a package from the Arch User Repository}
      \begin{itemize}
        \item For this example, we will install \verb|chromium-widevine| which
          is used to play Netflix videos in chromium.
        \item Begin by making a directory for downloading these packages.
\begin{verbatim}
$ mkdir ~/aur
$ cd ~/aur
\end{verbatim}
        \item Download the repository. Replace \verb|chromium-widevine| with
          the package name.
\begin{verbatim}
$ git clone https://aur.archlinux.org/chromium-widevine.git
$ cd ./chromeium-widevine
\end{verbatim}
        \item Build the package. This uses the \verb|PKGBUILD| file which is
          (semi) readable.
\begin{verbatim}
$ makepkg -si
\end{verbatim}
        \item
          Note that this will not automatically be updated. I'm working on
          scripts for this\ldots{}
      \end{itemize}

  \subsection{Install a Printer}
    \begin{itemize}
      \item Printer is managed by CUPS. Package \verb|cups-pdf| allows for print
        to pdf.
      \item The Epson WF-3520 printer driver is in the repository
        \verb|epson-inkjet-printer-escpr|.
\begin{verbatim}
$ sudo pacman -S cups cups-pdf epson-inkjet-printer-escpr
\end{verbatim}
      \item Using systemd, enable and start CUPS.
\begin{verbatim}
# systemctl enable org.cups.cupsd.service
# systemctl start org.cups.cupsd.service
\end{verbatim}
      \item Use the CUPS web/html interface to install the Epson WF-3520 and
        cups-pdf printers. Point a web browser to \verb|http://localhost:631|
        (port 631 on the localhost).
      \item When I did this, it automatically found the Epson printer on the
        network. Follow the prompts. I had to select the WF-3620 driver instead.
        It seems to work alright.
      \item When installing the cups-pdf printer, select ``Generic'' when
        prompted to select a make/brand.
      \item For cups-pdf, by default the pdf files are saved to
        \verb|/var/spool/cups-pdf/wcdawn|. You can change this by editing
        \verb|/etc/cups/cups-pdf.conf| and I set it to dump to my home
        directory. I can rename and move from there.
    \end{itemize}

    \subsubsection{What to do when a printer isn't working.}
      \begin{itemize}
        \item List all of the printers on the computer.
\begin{verbatim}
$ lpstat -t
\end{verbatim}
        \item Try disabling and re-enabling the printer. For a device named
          \verb|HomePrinter|.
\begin{verbatim}
# cupsdisable HomePrinter
# cupsenable HomePrinter
\end{verbatim}
        \item If everything is wrong, try deleting the printer.
\begin{verbatim}
# lpadmin -x HomePrinter
\end{verbatim}
      \end{itemize}

  \subsection{Install a scanner.}
    \begin{itemize}
      \item This uses the \verb|sane| package. SANE stands for Scanner Access
        Now Easy.
      \item It will also be useful to have \verb|ghostscript| installed but
        this is also required for \verb|epstopdf| in \LaTeX.
\begin{verbatim}
# pacman -S sane ghostscript
\end{verbatim}
      \item Now, \verb|scanimage -L| will list all available scanners. Mine
        (Epson) was found automatically.
      \item Use \verb|scanimage --help --device="DEVICENAME"| to list
        all of the options for the device.
      \item For this purpose, I have written a script to scan multiple page pdfs
        using fairly standard options.
    \end{itemize}

  \subsection{Configuring pacman.}
    \begin{itemize}
      \item Pacman is the arch linux package manager. It's configuartion file is
        located at \verb|/etc/pacman.conf|.
      \item In \verb|/etc/pacman.conf| uncomment the \verb|Color| line to
        enable pretty text formatting.
      \item Optionally, add a line with \verb|ILoveCandy| to turn the loading
        bar into a pacman eating dots.
    \end{itemize}

    \subsubsection{Useful Pacman Commands.}
      \begin{itemize}
        \item \verb|pacman -S NAME| to intall a package named NAME.
        \item \verb|pacman -Ss NAME| search the remote repositories for a
          package named NAME. The argument also acceps regular expressions.
        \item \verb|pacman -Qn| to list installed packages. \verb|pacman -Qm|
          to list packages installed from the AUR.
          \begin{itemize}
            \item Supply the \verb|-q| option to not output version numbers.
            \item Supply the \verb|-e| option (explicit) to only list packages
              explicitly downloaded. This is useful for outputing to a text file
              and installing with a script.
          \end{itemize}
        \item \verb|pacman -Qdt| lists truly orphaned packages.
        \item \verb|pacman -Rns NAME| to remove a package named NAME and all of
          its dependencies and all of its system config files.
      \end{itemize}

\section{Things I Know but haven't really figured out.}
  \begin{itemize}
    \item Tools:
      \begin{itemize}
        \item \verb|nnn| (terminal file browser)
        \item \verb|chrony| (for chron jobs / time automation)
        \begin{itemize}
          \item Luke Smith has a video about chron jobs.
          \item One of his chron jobs automatically downloads packgae updates
            which seems super useful.
        \end{itemize}
      \end{itemize}
    \item There is no \verb|ifconfig| support natively. Use \verb|ip addr|
      instead.
  \end{itemize}

\end{document}
